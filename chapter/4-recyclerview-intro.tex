\chapter{RecyclerView 介绍}

RecyclerView 最早在 Google I/O 2016 被提出,用来解决传统应用程序中使用 ListView 处理复杂的列表时带来的性能问题。RecyclerView 本身也是一个 ViewGroup,与其它常用的 ViewGroup,如 ScrollView、ViewPager、ListView 等在绘制流程中的行为是几乎一致的。只不过,RecyclerView 更加专注于在流式布局的场景下,对于多种复杂的子项进行合理的回收和复用,从而在不断滑动的过程中,增加已有组件的利用率,提高滑动时的性能和流畅度。

RecyclerView 不是单独的一个组件。由于处理各种子 View 的回收和复用,以及处理动画、数据绑定等操作的逻辑都非常复杂,因此将这些逻辑都分成独立的组件。整个 RecyclerView 家族主要由以下几个组件组成:

\begin{itemize}
    \item RecyclerView:父 View 本身,负责响应原有的 View 绘制流程,作为其它组件行为的发起者;
    \item LayoutManager:负责 RecyclerView 的测量和布局流程,安排子 View 的位置。有不同的实现方式,比如线性布局,网格布局等等。本文中主要考虑在 Feed 流场景下常见的线性布局 LinearLayoutManger;
    \item ItemAnimator:负责子 View 的显示,隐藏等动画。因为不涉及到核心的布局流程,所以本文不涉及到这方面的优化;
    \item Adapter:RecyclerView 最核心的数据管理类。所有的子 View 中的数据由 Adapter 来管理,并在合适的时机进行绑定,从而能够显示在屏幕上。Adapter 作为数据的存储器,能够通知 RecyclerView 数据发生了何种变化,从而通知 RecyclerView 进行适当的刷新操作。
\end{itemize}

接下来将通过对 RecyclerView 本身的测量、布局、滑动、数据处理流程来认识这些组件,并引出响应的优化点。

\section{RecyclerView 的测量}

对于一个 ViewGroup 来说,它的测量过程就是要知道自己所有的子 View 的综合属性。然而,如果 RecyclerView 本身无法知晓子 View 都有谁,那自然无法进行测量。和其它的布局(如 LinearLayout,FrameLayout 等)不同,其它的布局在创建的时候,无论是通过 XML 还是手动在代码中调用 addView() 来创建,都是已经知道子 View 的完整状态;而 RecyclerView 获取子 View 参数信息的手段是通过 Adapter。而 Adapter 最重要的数据准备过程是交给开发者来决定的。因此,在初次测量时,RecyclerView 拿不到任何子 View 的信息。这个时候,如果RecyclerView 在布局方向(垂直或水平)上的属性是固定的值,那么测量就会很简单,直接返回对应的值即可;如果是一个不确定的值(比如 WRAP\_CONTENT),那么会先尝试进行一次布局流程,然后再进行测量。通常情况下, RecyclerView 在布局方向上的长度都是一个固定的值,因为这样能够很大程度上减少重复测量的次数,从而提高滑动的性能。

\section{RecyclerView 的布局}

布局是 RecyclerView 最关键的流程。这里涉及的就是对Adapter的消费,布局其中的子 View,以及回收和复用发生的地方。布局操作的整个流程在方法 dispatchLayout() 中,里面的流程分为三步。这三步所做的事情如下:

\begin{itemize}
    \item 第一步:处理 Adapter 中提交的更新,同时保存当前子 View 的参数信息;
    \item 第二步:对于每一个子 View,执行最终的测量和布局流程,确定它们安放的位置。这个过程也是 RecyclerView 和 Adapter 交互最主要的过程 —— 进行数据绑定。
    \item 第三步:主要处理动画的流程。因为不涉及到布局上的性能优化,本文不进行讨论。
\end{itemize}

接下来,我们将对 RecyclerView 最主要的布局流程进行分析。首先,RecyclerView 会拦截子 View 对布局的请求。在 RecyclerView 布局的过程中,子 View 本身就已经将要被布局。因此 RecyclerView 认为在这个时刻任何子 View 的布局行为都是多余的。因此 RecyclerView 并不会允许子 View 在这个时候请求布局。直到第三步结束之后,才再次允许子 View 进行布局请求。实际上,在真实开发过程中,也非常不建议在 RecyclerView 的子 View 中手动进行布局请求,而是使用 Adapter 去通知的方式进行增量更新。这样会有更高的效率和更安全的执行流程。

接下来是最重要的布局流程。可以简单概括为:计算出起始锚点(通常是延布局方向的起始位置),并沿着布局方向决定每个 View 的行为。有可能是创建新的 View,也有可能是复用之前回收的 View;并给这些确定会显示在屏幕上的 View 进行数据绑定。这个过程中,创建和绑定的操作是交给 Adapter 来完成的。也就是说,开发者需要在自己的 Adapter 中完成创建和绑定的流程。因此,这个过程中我们能够进行一些特殊的操作。比如提前拿到 View 进行绑定,甚至利用空闲时间异步创建出 View 来避免真正用到 View 时才去创建而导致耗时增加;另外要强调的一点是,由于 View 本身不能具有 RecyclerView 需要的准确的参数信息(位置信息,状态变化标记位等),同时也不能被妥善地回收和复用。因此 RecyclerView 框架的做法是在 View 上包装一层 ViewHolder 来持有 View 的引用。对于 LayoutManager 来说,它直接操作的是 ViewHolder 而非 View,这样能够更好地对 View 进行回收和复用,也便于增加一些位置变化等关键信息来让 LayoutManager 更迅速地进行布局操作。

\section{RecyclerView 对于滑动过程的处理}

根据之前介绍的 View 对于用户输入事件的处理,我们应该明白

\section{RecyclerView 对数据改动的响应}