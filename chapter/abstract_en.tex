RecyclerView is the most important flow layout component in the Android development field. The majority of applications' Feed streams (Information Streams) on the market are developed based on RecyclerView. Feed streams are the core business scenes of many popular apps nowadays, such as Taobao, TikTok, WeChat, and others. The performance of information streams directly determines the overall user experience of these applications.

Currently, more and more internet products are adding short video features on mobile platforms, such as Kuaishou, Douyin, Meituan, Xigua Video, and Xiaohongshu. The short video section in these applications is usually set as the default startup guide scene, i.e., the Feed stream, and is the section with the longest average consumption time in the entire application. Therefore, the smooth experience of this part is crucial for user retention and consumption duration indicators. Being able to optimize the video start-up speed, scrolling smoothness, network request speed, and any other scenarios can bring significant business benefits.

This article focuses on performance optimization for Feed streams developed based on RecyclerView. The main optimization goal is to perform pre-rendering optimization for Feed streams with video scenes as the core, while minimizing the impact on the business side. At the same time, optimization techniques for other types of Feed streams should also be explored. For these optimization scenarios, a quantifiable metric is needed to validate the improvement in smoothness due to optimization efforts. By analyzing the fluency metrics, it has been determined that pre-rendering optimization for the data binding phase can bring significant fluency benefits. The more time-consuming the data binding logic, the more noticeable the optimization effect.

% Optimizing based on RecyclerView is of great significance for the overall performance improvement and even user growth of the entire Android application ecosystem. Since users spend almost all their time consuming content in the app's Feed stream, and the improved app's smoothness encourages continuous consumption rather than premature exits, even small performance improvements can yield significant benefits in the Feed scene. In this article, the pre-rendering optimization for video scenes completely bypasses the data binding phase during user scroll events. It can be concluded through quantifiable metrics that this optimization greatly enhances the user's smoothness experience.

% RecyclerView is the most important component for implementing a staggered layout in the field of Android development, and the majority of feed streams in popular apps on the market are developed based on RecyclerView. The feed stream, also known as a staggered information stream, is the core business scenario of many popular apps such as Taobao, TikTok, and WeChat etc. The performance of the information stream directly determines the overall user experience of these applications.

% RecyclerView is almost universally used in all modern Android apps, and its application scenarios are often the most complex business logic in an app. Therefore, developers' customization of RecyclerView (whether it's ViewHolder or data binding logic) is often the most complicated, code-heavy, and time-consuming part of an application. At the same time, developers' understanding and usage of RecyclerView directly impact the experience of a large number of users. At Google I/O 2016, the official reasons for the creation of RecyclerView were given, along with a rough overview of how RecyclerView works.

% However, relying solely on the information provided by Google does not effectively help us understand the internal principles of RecyclerView, thus making it difficult to better use or modify this complex staggered layout. More and more articles are constantly appearing, analyzing the source code of RecyclerView and dissecting its architecture, code paths, and algorithms.

% So far, both domestic and international Internet giants have not ceased their research on RecyclerView, and they continue to explore performance optimization points for specific business scenarios, aiming to optimize the startup, scrolling, and data loading time of RecyclerView.

% This paper will delve into the development and performance optimization techniques of RecyclerView, and develop a high-performance feed stream. Real-world online applications will be used to validate some of the performance optimization techniques.
