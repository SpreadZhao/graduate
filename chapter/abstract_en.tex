RecyclerView\cite{test} is the most important component for implementing a staggered layout in the field of Android development, and the majority of feed streams in popular apps on the market are developed based on RecyclerView. The feed stream, also known as a staggered information stream, is the core business scenario of many popular apps such as Taobao, TikTok, and WeChat etc. The performance of the information stream directly determines the overall user experience of these applications.

RecyclerView is almost universally used in all modern Android apps, and its application scenarios are often the most complex business logic in an app. Therefore, developers' customization of RecyclerView (whether it's ViewHolder or data binding logic) is often the most complicated, code-heavy, and time-consuming part of an application. At the same time, developers' understanding and usage of RecyclerView directly impact the experience of a large number of users. At Google I/O 2016, the official reasons for the creation of RecyclerView were given, along with a rough overview of how RecyclerView works.

However, relying solely on the information provided by Google does not effectively help us understand the internal principles of RecyclerView, thus making it difficult to better use or modify this complex staggered layout. More and more articles are constantly appearing, analyzing the source code of RecyclerView and dissecting its architecture, code paths, and algorithms.

So far, both domestic and international Internet giants have not ceased their research on RecyclerView, and they continue to explore performance optimization points for specific business scenarios, aiming to optimize the startup, scrolling, and data loading time of RecyclerView.

This paper will delve into the development and performance optimization techniques of RecyclerView, and develop a high-performance feed stream. Real-world online applications will be used to validate some of the performance optimization techniques.