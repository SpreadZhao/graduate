\chapter{项目介绍}

Android 是目前使用非常广泛的系统\cite{businge2019studying}。无论是手机、平板、电子书、车机还是嵌入式设备等都在运行 Android 系统。Android 应用程序的开发和迭代也的速度也在日益增加。随着代码不断累积,应用本身的业务逻辑也越来越臃肿。因此对于应用的主要业务场景,我们需要对其进行不断优化,才能削减业务代码增加带来的性能方面的负面影响。

Android 系统的大部分 UI 绘制是交由应用的主线程来执行\cite{yan2014real}。主线程从 ActivityThread 的 main 方法开始运行,通过消息机制来不断处理 UI 的绘制消息和应用内部在主线程执行的业务逻辑,从而让应用运行下去,并响应用户的操作行为而进行 UI 上的改变。在这个过程中,如果主线程进行了过多的操作导致消息不能及时处理,就会发生在主线程的卡顿,从而让 UI 不能及时刷新,影响用户的体验。因此,我们需要对主线程的代码进行不断优化和拆分,来让 UI 操作能够即使得到处理。

目前绝大部分 Android 应用的主要场景是一个信息流,用来给用户呈现最新的推荐信息。这些信息包括但不限于视频、新闻、帖子等流式信息。这些信息的刷新和展现大部分也都是由主线程来完成。而作为一个应用的主要场景,甚至是应用的门面,这些场景的业务代码通常更新频率非常高,业务逻辑迭代会非常快,导致这些场景的代码的劣化速度也远高于其它场景。所以,这部分代码对流畅性的影响是我们一定要尽可能削减的。

这些流式信息为了能够方便用户进行浏览,通常也都会用流式布局来承载。其中应用最广泛的就是 RecyclerView,一个高性能的流式布局容器\cite{sabiyath2020enhanced}。当用户在 RecyclerView 中快速滑动时,它会根据目前的情况合理分配和回收每一个子项目的引用和视图,从而减少该布局在内存和性能方面的开销。然而,随着业务代码的不断积累,RecyclerView 本身带来的性能收益也逐渐被淹没,同时 RecyclerView 也已经承载了太多的业务逻辑,很难在短时间内对其进行深度解耦。因此我们尝试基于它针对特定的业务场景进行深度优化,从而提高用户的流畅性体验。

显然,我们希望能够量化优化的结果。因此我们还需要一些手段来验证我们的优化结果。其中最复杂的部分就是如何找到采集指标信息的时机,并在这个时机进行信息的收集。最普遍的手段就是帧率指标,然而目前的帧率采集手段依然有一些不足。因此我们也要探究出一个更加准确的帧率采集手段,并在这个基础上,继续探索其它的性能指标来衡量我们对于 RecyclerView 的其它方面做出的优化成果。

最后,不只局限于视频类型的应用,RecyclerView 更多被应用于大量图文为基础的信息流。针对这种场景也需要进行一定程度的优化,并通过其它方面的量化指标来衡量优化的效果。