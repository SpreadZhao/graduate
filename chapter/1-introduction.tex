\chapter{绪论}

\section{研究背景及意义}

Android 是目前使用非常广泛的系统\cite{businge2019studying}。无论是手机、平板、电子书、车机还是嵌入式设备等都在运行 Android 系统。Android 应用程序的开发和迭代也的速度也在日益增加。随着代码不断累积,应用本身的业务逻辑也越来越臃肿。因此需要对应用的主要业务场景不断优化,才能削减业务代码增加带来的性能方面的负面影响。

Android 系统的大部分 UI 绘制是交由应用的主线程来执行\cite{yan2014real}。主线程从 ActivityThread 的 main 方法开始运行,通过消息机制来不断处理 UI 的绘制消息和应用内部在主线程执行的业务逻辑,从而让应用运行下去,并响应用户的操作行为而进行 UI 上的改变。在这个过程中,如果主线程进行了过多的操作导致消息不能及时处理,就会发生卡顿,从而让 UI 不能及时刷新,影响用户的体验。因此需要对主线程的代码进行不断优化和解耦,让 UI 操作及时得到处理。

目前绝大部分 Android 应用的主要场景是一个信息流,来给用户呈现最新的推荐信息。这些信息包括但不限于视频、新闻、帖子等流式信息。这些信息的刷新和展现大部分也都是由主线程来完成。而作为一个应用的主要场景,甚至是应用的门面,这些场景的业务代码通常更新频率非常高,业务逻辑迭代也会非常快,导致这些场景的代码的劣化速度也远高于其它场景。所以,这部分代码对流畅性的影响是要尽可能削减的。

这些流式信息为了方便用户浏览,通常也都会用流式布局来承载。其中应用最广泛的就是 RecyclerView,一个高性能的流式布局容器\cite{sabiyath2020enhanced}。当用户在 RecyclerView 中快速滑动时,它会根据目前的情况合理分配和回收每一个子项目的引用和视图,从而减少其在内存和性能方面的整体开销。然而,随着业务代码的不断积累,RecyclerView 本身带来的性能收益也会逐渐被淹没,同时 RecyclerView 也已经承载了太多的业务逻辑,很难在短时间内深度解耦。因此本文尝试从另一个角度针对特定的业务场景进行深度优化,从而提高用户的流畅性体验。

基于 RecyclerView 的优化对于整个 Android 应用的性能大盘提升甚至是用户增长都有很重要的意义。由于用户的消费时长几乎都在应用的 Feed 流中,并且应用本身的流畅性的提高也有利于用户持续进行消费而不是中途退出,因此即使是比较小的性能提升,折算在 Feed 场景中也能得到很大的收益。在本文中,具体开展的 Feed 流优化工作是针对数据绑定阶段的。在优化后,数据绑定阶段会尽可能避开用户触发的滚动事件,通过量化的指标可以观察到该优化极大地提升流用户的流畅性体验。

\section{国内外研究现状}

目前国内外对于 Android 应用流畅性优化的工具和检测手段有很多。以谷歌官方为例,在 RecyclerView 的渲染流程中,提供了用于数据异步加载的 AsyncListUtil 工具,用于增量刷新的 DiffUtil 工具,同时也在 RecyclerView 内部集成了很多性能优化的轻量组件。针对流畅性监控以及一些常见的性能优化场景,国内的很多互联网企业也贡献了许多开源的项目。如腾讯的 MMKV,快手的 KOOM,字节跳动的 RheaTrace 等等。合理地利用这些检测和优化工具,能够快速定位需要优化的场景,并迅速将一些常见的优化手段落地,从而评估整体的优化空间和收益。

目前针对应用 Feed 场景的公开优化手段非常有限,尤其是业务高度复杂的 Feed 流场景。而这部分场景恰恰也是用户感知最强烈,最能影响用户整体流畅性体验的场景。通过在实习单位中进行内部调研,发现这部分的优化对于一些用户量庞大、业务结构复杂的互联网应用来说是非常关键的,同时也已经有非常多的优化策略在内部流行。然而,以流畅性优化为例,这些对内的优化场景大多与应用本身的业务逻辑高度耦合,很难做到框架级别的封装。即使能够做到,对于 Android 设备的支持广度也相当有限。

\section{本文的主要工作介绍}

本文致力于针对 RecyclerView 数据绑定这个细化但核心的场景进行优化。设计一个轻量化的预渲染框架,能够让 RecyclerView 的数据绑定避开其它的耗时逻辑,从而优化用户在滑动过程中的整体流畅性。

为了能够量化优化的结果,还需要一些手段来验证。其中最复杂的部分就是如何找到采集指标信息的时机,并在这个时机进行信息的收集。最普遍的手段就是采集帧率指标,然而目前业界内的帧率采集手段依然有一些不足。所以也要探究出一个更加准确的帧率采集手段,并在这个基础上,继续探索其它的性能指标来衡量对于 RecyclerView 的其它方面做出的优化成果。

最后,不只局限于视频类型的应用,RecyclerView 更多被应用于以大量图文为基础的信息流。针对这种场景也需要进行一定程度的优化,并通过其它方面的量化指标来衡量优化的效果;同时也设计了一个秒开检测框架来收集相应的性能指标。

\section{本文的结构安排}

\begin{itemize}

    \item 相关技术与理论基础:由于业界并没有详细的针对 RecyclerView 的解析,同时本文的优化需要在其基础上对核心部分进行修改,所以为了避免对业务方产生影响,需要对 RecyclerView 以及 Android 整体的绘制体系进行深度调研。在有了这些知识储备的情况下,才能尽可能全面地排查出优化中产生的副作用。本部分对应第 2 章。

    \item Feed 流优化工作:接下来,会介绍预渲染框架具体应用场景的背景、预渲染技术方案的选型、框架的总体设计思路和细节实现、开发过程中遇到的问题和解决过程。同时该项目也已经在实习公司内经过一轮完整的需求提出、需求评估、需求开发、缺陷修复、灰度测试、线上通过实验验收的流程。同时,针对一些以图文为主的 Feed 流场景的流畅性优化,也进行了简单的调研和开发。本部分对应第 3 章。

    \item Feed 流优化验证:最后,是对于这些优化手段的验证。通过设计的框架收集性能指标,从而验证优化的结果。同时,为了能够模拟真实的 Feed 流场景,还开发了一个简单的 Feed 流用作优化目标。本部分对应第 4 章。

\end{itemize}
