RecyclerView 是安卓开发领域最重要的流式布局组件,市面上大多数的应用的 Feed 流(信息流)都是基于 RecyclerView 进行开发。Feed 流是目前很多流行的大众APP的核心业务场景,如淘宝、抖音、微信等。信息流的性能表现,直接决定着这些应用的整体用户体验。

本文针对基于 RecyclerView 开发的 Feed 流开展性能优化工作。主要的优化目标为以视频场景为核心的 Feed 流,在对业务方影响尽可能小的情况下进行预渲染优化;同时对于其它类型的 Feed 流,也要探究出优化的手段。对于这些优化场景,需要给出一个可以量化的指标来验证优化对于流畅性的提升。

基于 RecyclerView 的优化对于整个 Android 应用的性能大盘提升甚至是用户增长都有很重要的意义。由于用户的消费时长几乎都在应用的 Feed 流中,并且应用本身的流畅性的提高也有利于用户持续进行消费而不是中途退出,因此即使是比较小的性能提升,折算在 Feed 场景中也能得到很大的收益。在本文针对视频场景进行的预渲染优化中,能够将数据绑定阶段完全避开用户的滚动事件。通过量化的指标也可以总结出,该优化极大地提升了用户的流畅性体验。

% RecyclerView几乎在所有现代Android App中都有所使用,并且应用场景往往是一个App最复杂的业务。因此开发者对于RecyclerView的自定义(无论是ViewHolder,还是数据绑定逻辑)往往是一个应用中逻辑最复杂,代码量最多,耗时最长的部分;同时开发者对于RecyclerView的掌握程度以及使用方式直接影响着大量用户的体验。在Google I/O 2016上,官方给出了RecyclerView产生的原因,并且给出了大致的RecyclerView工作方式。

% 然而,仅仅依靠Google给出的资料并不能让我们有效理解RecyclerView的内部原理,从而更好地使用甚至修改这个复杂的流式布局。越来越多的文章不断出现,解析RecyclerView内部的源码,并对其中的架构设计、代码链路、算法进行分析。

% 目前为止,国内外互联网大厂对于RecyclerView的研究依然没有停止,并且针对特定的业务场景,依然在不断寻找性能优化点,优化RecyclerView的启动、滑动、数据加载耗时。

% 本课题将深入研究RecyclerView的开发与性能优化手段,开发出一个高性能的Feed流,并基于真实的线上应用去验证一些性能优化手段。