RecyclerView\cite{test}是安卓开发领域最重要的流式布局组件,市面上大多数的APP的Feed流都是基于RecyclerView进行开发。Feed流也就是流式信息流,目前很多流行的大众APP的核心业务场景都是一个Feed流,如淘宝、抖音、微信等。信息流的性能表现,直接决定着这些应用的整体用户体验。

RecyclerView几乎在所有现代Android App中都有所使用,并且应用场景往往是一个App最复杂的业务。因此开发者对于RecyclerView的自定义(无论是ViewHolder,还是数据绑定逻辑)往往是一个应用中逻辑最复杂,代码量最多,耗时最长的部分;同时开发者对于RecyclerView的掌握程度以及使用方式直接影响着大量用户的体验。在Google I/O 2016上,官方给出了RecyclerView产生的原因,并且给出了大致的RecyclerView工作方式。

然而,仅仅依靠Google给出的资料并不能让我们有效理解RecyclerView的内部原理,从而更好地使用甚至修改这个复杂的流式布局。越来越多的文章不断出现,解析RecyclerView内部的源码,并对其中的架构设计、代码链路、算法进行分析。

目前为止,国内外互联网大厂对于RecyclerView的研究依然没有停止,并且针对特定的业务场景,依然在不断寻找性能优化点,优化RecyclerView的启动、滑动、数据加载耗时。

本课题将深入研究RecyclerView的开发与性能优化手段,开发出一个高性能的Feed流,并基于真实的线上应用去验证一些性能优化手段。