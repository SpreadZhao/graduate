\chapter{预渲染框架的背景}

在视频 Feed 流中,对于下一个视频信息的加载过程通常是需要格外小心的。因为该过程发生在用户滑动屏幕的时候,此时主线程正不断接收用户发送的滑动事件,并根据这些滑动的事件来进行布局、UI 刷新等操作,占据了主线程很大一部分资源。如果这个时候数据绑定的流程过于复杂,会严重增加这段时间的主线程耗时,从而影响整体的流畅性,甚至造成肉眼可见的卡顿。因此,在 Adapter 进行数据绑定的过程中,是强烈不建议在主线程进行任何耗时任务的。这些任务要么通过异步的方式后续补齐,要么针对具体的业务,寻找合适的时机提前进行。

然而,经过业务的不断迭代,数据绑定的耗时一定会随着业务方不断添加新的功能而增加。因此,我们希望能够对这个过程进行彻底的优化。在视频 Feed 流滑动的过程中,通常用户最希望看到的是下一个视频的封面立刻展示在屏幕上。为了达到这样的目标,通用的方案有针对图片、视频参数等数据做异步加载,同时在播放器的方面也进行一定程度的预渲染。但是由于 RecyclerView 整体的刷新流程是固定的,所以数据绑定是一个必要的过程。不过在对 RecyclerView 进行了足够多的调研之后,我们可以着手进行更加深度的优化,也就是能够在一个合适的时机将整个数据绑定操作提前进行,而不是只异步化其中的一小部分。在视频 Feed 流的场景中,合适的时机是比较容易寻找的,因为用户在滑动到一个视频之后通常会停留一段时间进行交互行为。但是这样的行为会让 RecyclerView 的生命周期产生混乱,因此即使能够做到,也需要经过细致的修复来让其能够准确地派发原先的生命周期。