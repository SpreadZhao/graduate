\chapter{Android 消息机制}

Android 应用程序运行的基本框架就是基于消息机制,同时绘制行为和消息机制的配合也非常紧密。Android 系统应用内部的消息机制主要和这几个类相关:Looper、Handler 和 MessageQueue。

Android 系统的主线程对应的类是 ActivityThread。该类内部维护着主线程启动,并处理各个线程发送消息的流程,同时也负责 UI 的绘制消息的处理。当该线程启动时,同时会准备自己的 Looper 来负责消息队列的管理和任务的委派。ActivityThread 中的 Looper 成为 Main Looper。由于只能由主线程来处理 UI 类型任务的分发,所以 Main Looper 也只允许主线程持有。如果应用内部的线程也想利用这套消息机制来进行任务处理(比如埋点上报的线程),也可以准备自己的 Looper 来接收任何线程发送给当前线程的任务并添加到消息队列中,在某一个时刻被取出并执行。

Looper 内部管理的消息队列的实现类就是 MessageQueue。其内部管理了任务队列的管理操作,包括入队、出队,以及空闲时的敏捷任务处理操作等。大部分是通过调用 Android Native 的 C++ 代码来实现,并通过 JNI 包装成面向 Android Framework 以及 Android 应用程序的接口。

Android 程序中,消息不能直接通过访问 Looper 或者 MessageQueue 的方式进行提交,只能通过封装好的方式 —— Handler 进行提交。这种方式更加安全和方便,也能给应用的架构设计提供更好的开发范式。在任何线程中都可以创建 Handler,但是 Handler 在构造的时候必须指定 Looper,也就是明确提交的任务需要由哪一个线程的 Looper 来管理。同时,Handler 也提供了多种任务执行的方式、多种提交任务的方式以及各种延时和取消的功能。例如,我们可以在发送一个任务之前,取消之前已经提交但未执行的特定类型的任务来满足特定的业务需求。在对 RecyclerView 的布局流程进行改造时,我们也会利用到这一特性。