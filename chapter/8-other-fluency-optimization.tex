\chapter{其它形式的优化手段}

\section{View 的异步创建}

Feed 流本身并不局限于视频场景。如微信、今日头条、美团等应用的 Feed 流更多程度上是图文形式、非粘性滑动、具有大量内容的 RecyclerView。在这些场景下,预渲染的思路本身就不适合引入,因为这些场景下屏幕上同时会有非常多的卡片,并且由于非粘性滑动,用户的每一次滑动都不是准确的定位到一个特定的卡片。因此,对于这些场景,更多的优化手段是让首次刷新更加迅速、让卡片创建的耗时尽可能降低等等。本文中主要采用的优化手段为将卡片进行异步创建(Inflate)。

在正常的 RecyclerView 创建卡片时,都是通过 Adapter 的回调实现卡片创建的逻辑。在内部通常会使用布局加载器 LayoutInflater 来通过 XML 文件加载 View。这种方式最耗时的任务就是对于 XML 文件的解析,并根据文件中的 View 层级创建出 View 实例的操作。如果将这部分流程提前进行,在用到的时候直接从缓存池中取用,就能够极大程度地减少 View 创建的耗时。谷歌官方已经提供了将 View 进行异步创建的组件 —— AsyncLayoutInflater。该组件内部拥有一个异步线程以运行业务方提交的 View 加载任务。但是异步创建 View 的难点并不是使用组件进行加载,而是找到加载的时机和用于保存加载出的 View 的缓存。

当 RecyclerView 被初始化时,内部还没有数据。需要等 Adapter 初始化完毕,并主动通知 RecyclerView 数据发生了变化之后,RecyclerView 才会进行相应的布局操作,并在这个过程中将子 View 初始化。在这个过程中,数据到达通常需要等待一段网络 IO 的时间。因此,在这个时间间隔内,我们可以用来进行 View 的异步创建。并且由于 RecyclerView 展示的子 View 的属性通常都是一致的,因此我们可以提前知晓 View 的具体类型,并通知异步的 AsyncLayoutInflater 进行加载;View 的缓存选用 SparseArray。因为 SparseArray 与 HashMap 有相似的结构,所以通过 View 的 id 去访问通常有更高的性能;同时如果需要大量的读写操作,SparseArray 也与队列等结构有相似的效率,比数组在移除、添加操作上有更高的效率。

\section{异步创建优化验证}

为了验证异步创建 View 优化的效果,搭建了一个类似新闻页面的 RecyclerView。此页面在首次刷新时会进行一次假的网络请求,在一段时间之后会返回一些新闻数据,同时用于 Adapter 进行数据绑定并通知 RecyclerView 进行布局。为了模拟 View 创建过程中的耗时行为,人为在子 View 的构造方法中引入耗时逻辑。在这个过程中,开启异步预加载和关闭预加载的情况下,可以明显感觉到有异步创建的时候加载的速度更快。为了进一步量化优化效果,设计了一个“秒开检测框架”:当 RecyclerView 首次刷新时,通过 ViewTreeObserver 的监听器捕获事件,并在回调方法中执行检测逻辑。

秒开框架的检测思路如下:将屏幕分成若干个目标区域。每个区域通过左上角顶点的坐标,以及区域的长和宽来唯一标识。当进行检测时,扫描检测范围内所有的 View,让业务方来决定有效区域的大小。这里的有效区域是指对用户来说有意义的区域大小。以文本类 TextView 来举例子,有效区域就是真实文本所覆盖的区域,不包括四周的留白。当有效区域和屏幕的目标区域的交集超过了一定阈值时,该目标区域被标记为有效;当所有(或大部分)的目标区域都被认为有效时,此时的时刻为页面首次刷新的时长,及秒开时长。该时长标志着应用加载的速度以及给用户带来的体感。时间越短表明对用户来说刷新的感知时间越短,体验越好。

当没有异步创建,以及有异步创建的情况下缓存容量逐渐增加的过程中,缓存个数与秒开时长的关系如下:

这代表如果我们可以针对具体的业务背景选择合适的缓存大小,从而获得最高的首刷流畅性体验。